My original background is Materials Engineering by Universidade Federal de Santa Catarina - UFSC (2011), Florianópolis, Brazil, next I received my PhD in Materials Science by Université de Lorraine - UL (2017), Nancy, France. Early in my career I have acquired a good knowledge of technical standards, management and manufacture practices, quality management, and developed good interpersonal skills, especially in production settings. Later I accumulated skills in laboratory and production data treatment and analysis for scale-up of industrial processes. Also successfully applied numerical simulation to the thermal and thermochemical processing of materials: carburizing, nitriding, selective oxidation, ceramics, etc. This was possible thanks to the skills developed during my thesis years (2013-2017) that led me to the positions I held at ArcelorMittal, the world's largest steelmaker, then at Imerys, leader in specialty minerals.
%
\par\vskip6pt%
%
Unknowingly, I started with Machine Learning while applying semi-supervised graph methods for simplification of chemical systems to allow the simulation of vacuum carburizing of steel early in my graduate studies and my love by the field simply continued growing. That later opened the path to an important moment of my career, when I led the digital transformation in a team of about 70 people. That role has proven important in my personal growth and interpersonal skills, as I have interacted, proposed solutions, and received feedback from all levels. Although I had great pleasure with the digital world, that led me too far away from my core passions related to materials, which I aim to rejoin in the near future. Obviously, the skills I have acquired have changed the way I work towards a more efficient and organized way, something I hope to continue improving.

\endinput%