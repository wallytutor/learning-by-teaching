My original background is Materials Engineering by Universidade Federal de Santa Catarina - UFSC (2011), Florianópolis, Brazil, next I received my PhD in Materials Science by Université de Lorraine - UL (2017), Nancy, France. Early in my career I have acquired a good knowledge of technical standards, management and manufacture practices, quality management, and developed good interpersonal skills, especially in production settings. Later I accumulated skills in laboratory scale and process data treatment and analysis, comfortably employing the major open-source tools in the field, such as Python and Julia programming. Also successfully applied numerical simulation (CFD) to the thermal and thermochemical processing of materials. This was possible thanks to my thesis years (2013-2017) that allowed me to develop a broad range of computational skills, what came to be the core of the positions I occupied at ArcelorMittal, the world's largest steel producer, then at Imerys, world's leader in specialty minerals.
%
\par\vskip6pt%
%
Unknowingly, I started with Machine Learning while applying semi-supervised graph methods for simplification of chemical systems early in my graduate studies and my love by the field simply continued growing. That later opened the path to what I consider the most important moment of my career, when I led the digital transformation in a team of about 70 people. That role has proven important in my personal growth and interpersonal skills, as I have been interacting, proposing solutions, and receiving feedback from all levels. Those skills inspired me to turn my career towards innovation and technical coaching of engineers in the fields of process modeling and simulation, applied technical machine learning, and other Data Science and Machine Learning related disciplines.

\endinput%