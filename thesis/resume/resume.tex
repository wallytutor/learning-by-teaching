\documentclass[10pt]{article}
\usepackage[frenchb]{babel}
\usepackage{geometry}
\usepackage{fontspec}

\geometry{tmargin=2cm, bmargin=2cm, lmargin=2cm, rmargin=2cm}
\setmainfont{DejaVu Sans}
\pagestyle{empty}

\begin{document}
\noindent\emph{Mise au point de la carbonitruration gazeuse des alliages
16NiCrMo13 et 23MnCrMo5: modélisation et procédés}\par\vskip0.4cm

\noindent\textbf{Résumé:} Le développement de matériaux d'ingénierie combinant
ténacité et résistance à l'usure reste encore un défi. Dans le but de contribuer
à ce domaine, cette thèse présente une étude de la carbonitruration des aciers
16NiCrMo13 et 23MnCrMo5. L'évolution cinétique des atmosphères à base
d'hydrocarbures et d'ammoniac est étudiée numériquement, ainsi que le
comportement local à l'équilibre et la cinétique de diffusion pour l'obtention
de profils d'enrichissement des alliages traités. Les simulations sont
confrontées à des mesures par chromatographie en phase gazeuse des produits de
pyrolyse de l'acétylène et de décomposition de l'ammoniac, et aux réponses
métallurgiques, par l'évaluation des profils de diffusion, des filiations de
dureté et par l'identification des précipités formés par microscopie
électronique en transmission. La dureté obtenue après trempe et traitement
cryogénique évolue selon la racine carrée de la teneur en interstitiels en
solution solide simulée à partir de la composition locale en utilisant des
mesures des profils chimiques en carbone et en azote. Après revenu, les zones
enrichies en azote montrent une tenue en dureté supérieure à celles obtenues
avec la même teneur totale en carbone en solution, ce qui a été attribué après
observation par microscopie électronique en transmission à une fine
précipitation de nitrures de fer lors de cette dernière étape de traitement. Le
bilan de matière des produits de pyrolyse montre que les principales espèces non
détectées sont des radicaux fortement carbonés qui peuvent aussi donner lieu à
la formation d'hydrocarbures polycycliques de haut poids moléculaire dans les
zones froides du réacteur. À la pression atmosphérique et à basse pression
l'établissement de conditions d'enrichissement en carbone à concentration
constante est possible en utilisant de faibles pressions partielles d'acétylène
dilué dans l'azote. La conversion atteinte par la pyrolyse de ce précurseur est
pourtant importante à la température de traitement compte tenu du temps de
séjour caractéristique du réacteur employé à la pression atmosphérique. La
cinétique de décomposition de l'ammoniac étant beaucoup plus lente que celle des
hydrocarbures légers, il a été possible de quantifier la vitesse de
décomposition de cette espèce par unité de surface métallique exposée pendant la
durée d'un traitement.

\par\vskip0.4cm\noindent\textbf{Mots-clés:} Carbonitruration; Traitements
thermochimiques; Martensite; Cinétique chimique; Modèles de réacteur.

\par\vskip0.4cm \noindent\emph{Development of gas carbonitriding of alloys
16NiCrMo13 and 23MnCrMo5: modeling and processes}\par\vskip0.4cm

\noindent\textbf{Abstract:} The development of engineering materials combining
both toughness and wear resistance is still a challenge. Aiming to contribute to
this field of study, this thesis presents a study of the carbonitriding process
of alloys 16NiCrMo13~and 23MnCrMo5. Kinetics of hydrocarbon- and ammonia-based
atmospheres, as well as local equilibrium and diffusion kinetics for achieving
the enrichment profiles, are studied by numerical simulation. These simulations
are compared to chromatography measurements of gas phase pyrolysis products of
acetylene and ammonia decomposition, and with metallurgical responses, where the
comparison is made with evaluated diffusion profiles, hardness measurements and
the identification of precipitates by transmission electron microscopy. Hardness
after quench and cryogenic treatment depends on the square root of total solid
solution interstitial content simulated by using local carbon and nitrogen
compositions obtained experimentally. After tempering, the regions enriched in
nitrogen show better hardness stability than those with same total carbon
interstitial content, what was linked to a fine precipitation of iron nitrides
observed by transmission electron microscopy. Mole balance of pyrolysis products
show that the main non-detected species are high-carbon radicals, which may also
lead to the formation of polycyclic aromatic hydrocarbons of high molecular
weight at the reactor outlet. At both atmospheric and reduced pressures,
constant concentration enrichment boundary conditions were established by using
low partial pressures of acetylene diluted in nitrogen. Pyrolysis of this
precursor attains high conversion rates at treatment conditions given the
important residence time of the atmospheric pressure reactor. Ammonia
decomposition kinetics being much slower than that of low molecular weight
hydrocarbons, it was possible to identify the decomposition rate of this species
over a metallic sample during  a treatment. 

\par\vskip0.4cm\noindent\textbf{Keywords:} Carbonitriding; Thermochemical
treatments; Martensite; Chemical kinetics; Reactor models.

\end{document}
